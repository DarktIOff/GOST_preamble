\usepackage{extsizes} % Для добавления в параметры класса документа 14pt

% Для работы с несколькими языками и шрифтом Times New Roman по-умолчанию
\usepackage[english,russian]{babel}
\usepackage{fontspec}
\setmainfont{Times New Roman}

% ГОСТовские настройки для полей и абзацев
\usepackage[left=30mm,right=10mm,top=20mm,bottom=20mm]{geometry}
\usepackage{misccorr}
\usepackage{indentfirst}
\usepackage{enumitem}
\setlength{\parindent}{1.25cm}
\linespread{1.3}
\setlist{nolistsep} % Отсутствие отступов между элементами \enumerate и \itemize

% Дополнительное окружения для подписей
\usepackage{array}
\newenvironment{signstabular}[1][1]{
	\renewcommand*{\arraystretch}{#1}
	\tabular
}{
	\endtabular
}


% Переопределение стандартных \section, \subsection, \subsubsection по ГОСТу;
% Переопределение их отступов до и после для 1.5 интервала во всем документе
\usepackage{titlesec}

\titleformat{\section}[block]
{\bfseries\normalsize\filcenter}{\thesection}{1em}{}

\titleformat{\subsection}[hang]
{\bfseries\normalsize}{\thesubsection}{1em}{}
\titlespacing\subsection{\parindent}{\parskip}{\parskip}

\titleformat{\subsubsection}[hang]
{\bfseries\normalsize}{\thesubsubsection}{1em}{}
\titlespacing\subsubsection{\parindent}{\parskip}{\parskip}

% Работа с изображениями и таблицами; переопределение названий по ГОСТу
\usepackage{caption}
\captionsetup[figure]{name={Рисунок},labelsep=endash}
\captionsetup[table]{singlelinecheck=false, labelsep=endash}

\usepackage{graphicx}
%\usepackage{slashbox} % Диагональное разделение первой ячейки в таблицах

% Цвета для гиперссылок и листингов
\usepackage{color}


%\usepackage{xr-hyper}
%\externaldocument{Cosmologyessentials, Introduction}
% Гиперссылки \toc с кликабельностью
%\usepackage{xr-hyper}
%
%\makeatletter
%\newcommand*{\addFileDependency}[1]{
%  \typeout{(#1)}
%  \@addtofilelist{#1}
%  \IfFileExists{#1}{}{\typeout{No file #1.}}
%}
%\makeatother
%
%\newcommand*{\myexternaldocument}[1]{
%    \externaldocument{#1}
%    \addFileDependency{#1.tex}
%    \addFileDependency{#1.aux}
%}

\usepackage{hyperref}

\hypersetup{
	linktoc=all,
	linkcolor=black,
	colorlinks=false,
}

% Листинги
\setsansfont{Arial}
\setmonofont{Courier New}

\usepackage{color} % Цвета для гиперссылок и листингов
\definecolor{comment}{rgb}{0,0.5,0}
\definecolor{plain}{rgb}{0.2,0.2,0.2}
\definecolor{string}{rgb}{0.91,0.45,0.32}

\usepackage{listings}
\lstset{
	basicstyle=\footnotesize\ttfamily,
	language=[Sharp]C, % Или другой ваш язык -- см. документацию пакета
	commentstyle=\color{comment},
	numbers=left,
	numberstyle=\tiny\color{plain},
	numbersep=5pt,
	tabsize=4,
	extendedchars=\true,
	breaklines=true,
	keywordstyle=\color{blue},
	frame=b,
	stringstyle=\ttfamily\color{string}\ttfamily,
	showspaces=false,
	showtabs=false,
	xleftmargin=17pt,
	framexleftmargin=17pt,
	framexrightmargin=5pt,
	framexbottommargin=4pt,
	showstringspaces=false,
	inputencoding=utf8x,
	keepspaces=true
}

\DeclareCaptionLabelSeparator{line}{\ --\ }
\DeclareCaptionFont{white}{\color{white}}
\DeclareCaptionFormat{listing}{\colorbox[cmyk]{0.43,0.35,0.35,0.01}{\parbox{\textwidth}{\hspace{15pt}#1#2#3}}}
\captionsetup[lstlisting]{
	format=listing,
	labelfont=white,
	textfont=white,
	singlelinecheck=false,
	margin=0pt,
	font={bf,footnotesize},
	labelsep=line
}


% Годные пакеты для обычных действий
\usepackage{ulem} % Нормальное нижнее подчеркивание
\usepackage{hhline} % Двойная горизонтальная линия в таблицах
\usepackage[figure,table]{totalcount} % Подсчет изображений, таблиц
\usepackage{rotating} % Поворот изображения вместе с названием
\usepackage{lastpage} % Для подсчета числа страниц
\bibliographystyle{ugost2008} % ГОСТ для списка литературы
\usepackage{amsmath} % для всякого математического
\usepackage{gensymb}
%\usepackage[toc,page]{appendix}
\usepackage[title]{appendix}
\usepackage[caption=false]{subfig}
\usepackage{etoolbox}
\usepackage{zref-totpages}
\usepackage{lipsum}

% для расчета количества
\usepackage{totcount}
%цитирований
\newtotcounter{citnum}
\def\oldbibitem{} \let\oldbibitem=\bibitem
\def\bibitem{\stepcounter{citnum}\oldbibitem}

\newcounter{totfigures}
\newcounter{tottables}

\providecommand\totfig{} 
\providecommand\tottab{}

\makeatletter
\AtEndDocument{%
  \addtocounter{totfigures}{\value{figure}}%
  \addtocounter{tottables}{\value{table}}%
  \immediate\write\@mainaux{%
    \string\gdef\string\totfig{\number\value{totfigures}}%
    \string\gdef\string\tottab{\number\value{tottables}}%
  }%
}
\makeatother

\pretocmd{\chapter}{\addtocounter{totfigures}{\value{figure}}\setcounter{figure}{0}}{}{}
\pretocmd{\chapter}{\addtocounter{tottables}{\value{table}}\setcounter{table}{0}}{}{}


\newcommand{\parder}[2]{\frac{\partial {#1}}{\partial {#2}}}
\newcommand{\sparder}[2]{\frac{\partial^{2} {#1}}{\partial {#2}^{2}}}
\newcommand{\dalambert}[1]{\Box{#1}}
\newcommand{\abs}[1]{\left|{#1}\right|}
